%  A simple AAU report template.
%  2015-05-08 v. 1.2.0
%  Copyright 2010-2015 by Jesper Kjær Nielsen <jkn@es.aau.dk>
%
%  This is free software: you can redistribute it and/or modify
%  it under the terms of the GNU General Public License as published by
%  the Free Software Foundation, either version 3 of the License, or
%  (at your option) any later version.
%
%  This is distributed in the hope that it will be useful,
%  but WITHOUT ANY WARRANTY; without even the implied warranty of
%  MERCHANTABILITY or FITNESS FOR A PARTICULAR PURPOSE.  See the
%  GNU General Public License for more details.
%
%  You can find the GNU General Public License at <http://www.gnu.org/licenses/>.
%
\input{setup/preamble.tex}% package inclusion and set up of the document
\input{setup/hyphenations.tex}% 
\input{setup/macros.tex}% my new macros
\theoremheaderfont{\normalfont\bfseries}
\theorembodyfont{\normalfont}
\theoremstyle{break}
\def\theoremframecommand{{\color{gray!50}\vrule width 5pt \hspace{5pt}}}

\newshadedtheorem{defini}{Definition}[chapter]
\newenvironment{definition}[1]{%
    \begin{defini}[#1]
}{%
    \end{defini}
}


\begin{document}
%frontmatter
\pagestyle{empty} %disable headers and footers
\pagenumbering{roman} %use roman page numbering in the frontmatter
\input{sections/summary_frontpage.tex}
\cleardoublepage
\pdfbookmark[0]{Contents}{label:contents}
\pagestyle{fancy} %enable headers and footers again
%mainmatter
\pagenumbering{arabic} %use arabic page numbering in the mainmatter

Vulnerabilities are being found all the time in software.
As new software and features get published, potential vulnerabilities get released.
The attacker has the advantage in terms of variety of attacks.
He can comfortably attempt each attack in his arsenal while the publisher frantically tries to patch up the breaches.
The publisher often finds out about a vulnerability when it is too late and important data has been stolen.
So the element of surprise is certainly also there.
As Bruce Schneier said:
\begin{quote}
`\textit{`Attackers are at an advantage in cyberspace – this will not always be true, but it’ll certainly be true for the next bunch of years – and that makes defense difficult.}''\cite{schneier_interview}  
\end{quote}

\paragraph{Vulnerabilities in web applications}
There are constantly popping up new technologies and old ones are evolving and developing new features.
These new technologies and features can be used by the attackers as well.
Therefore it is important to be updated on current and new technologies when developing anything to the web.\cite{web_security_importance}

The \citet{OWASP10} lists the most critical web application security flaws.
The list is produced by a broad spectrum of security experts. 
They recommend that each application is checked for these ten critical security flaws and each organisation gets aware of how to detect and prevent these flaws.

\paragraph{}
Considering the difficulty of these problems and the size of code bases in the average software project, it would be an advantage to have a tool that could help finding security vulnerabilities.
As both of us have an interest in Python development we decided to look into tools the could help developing secure Python web applications.
We encountered two Python web frameworks, Django and Flask, that both do their part in helping the developer.
But even though the framework helps the developer, the developer can still circumvent or overlook these features, so we looked into tools that could analyse and find vulnerabilities in Python code.
When examining this type of tools we did not find anything that satisfied our needs.
This report will thus create a tool that aids detection of security flaws in Python web applications, focusing on the Flask web application framework.

\paragraph{Result}
This report engaged in creating a tool for detecting security flaws in Python web applications.
Firstly, the background for the project was examined, including the Python programming language and the Flask web framework.
We then proceeded to manufacture some Flask applications with vulnerabilities.
The resulting applications would then be used to check if the tool was able to detect them.

The theory needed to build such a tool was then described.
The theory includes control flow analysis, the monotone framework and taint analysis.

The tool \pyt{} was then implemented and evaluated.
\pyt{} was able to find all the manufactured vulnerabilities.
It was also able to run on open source Flask projects, but was not able to find any vulnerabilities in these.

\printbibliography
\end{document}
