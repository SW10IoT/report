\section{Limitations}
\subsection{Dynamic Features}
The following dynamic features are not supported meaning that the right flow of the source code is not captured and therefore is not taken into account in the analysis.
\pyt{} does not support the built in functions \texttt{eval} and \texttt{exec}.
These functions take a string as input and are executed as Python.
This means that one has no idea of what can happen when these functions are used.

\pyt{} does also not support monkey patching.
Monkey patching is used to swap or change functions in order to manipulate modules or classes.
This is powerful but also difficult to capture.

Not supporting the above dynamic features means that taint can be introduced without \pyt{} noticing.
One way to fix this is to overtaint by just tainting all nodes containing \texttt{eval} and \texttt{exec} for instance.

Limitations should contain:
\begin{itemize}
\item ``Having no dynamic analysis is a major disadvantage'' - Rene
\item Monkey patching
\item Decorators(?)
\item Inbuilt/library (?)
\item Others?
\end{itemize}

\subsection{Language Constructs}
This section contains a list of language constructs that were not implemented due to time constraints and were not deemed important as most of them were not faced in the projects we ran as input.
Language constructs not implemented:
\begin{itemize}
\item \textbf{Async} - used for asynchronous programming.\cite{python_async}
\item \textbf{Decorators} - used for function transformations, explained in \cref{python:decorators}.
\item \textbf{Try} - used for catching exceptions.\cite{python_exception}
\item \textbf{Delete} - a statement to delete a name.\cite{python_delete}
\item \textbf{Assert} - a statement that can be used to insert debugging assertions into a program.\cite{python_assert}
\item \textbf{Global} - a statement that declares a name global so that one can access global names from other scopes.\cite{python_global}
\item \textbf{Yield} - used in generator functions to ``return'' one item at a time.\cite{python_yield}
\item \textbf{Starred} - used for unpacking (only partially implemented).\cite{python_unpacking}
\end{itemize}

