\section{Framework adaptor}\label{framework_adaptor}
Our goal of our tool is to target web application frameworks.
As there exist different frameworks and these are organised in different manners, we need to accommodate for these variations.
The framework adaptor is purposed to accommodate for different web application frameworks.
The \texttt{FrameworkAdaptor} class is an abstract class with an abstract method \texttt{run}.
It contains a list of CFGs where the adaptor is expected to put the result of the adaptation.
The implementation of \texttt{run} should include processing for all framework specific details that are needed in order to get a correct model of the application.
This adaptor can be implemented for an arbitrary framework like Django or, in our case, Flask.

\paragraph{The Flask Adaptor}
In Flask all web pages are defined in a function with a decorator.
This function is never explicitly called in the source code, so a single traversal by the CFG visitor of the file will not include all these methods in the resulting list of CFGs.
The \texttt{FlaskAdaptor} was implemented for this purpose.
The result of the CFG visitor contains all functions found during the traversal.
The Flask adaptor goes through these and finds all functions that are defined with the \texttt{@app.route()} decorator, and adds it to the list of CFGs.
This results in a list of CFGs that cover all the code needed for analysing the application.

\paragraph{Implementing a custom adaptor}
The adaptor component is in place for making it possible to adapt the analysis to any framework.
This paragraph will describe how such an implementation should be devised.

In \cref{appendix:flask_adaptor} the implementation of the Flask adaptor is shown.
This implementation will be used as a basis for this description.
The adaptor inherits from the class \texttt{FrameworkAdaptor} which contains an instance variable for a CFG list and the abstract function \texttt{run}.
The \texttt{run} function is the entry point to an adaptor.

In the Flask implementation the CFG list, which is initially populated with the result of the CFG visitor, is iterated over and the \texttt{find\_flask\_route\_functions} function is called on each. \footnote{In our case there is always one CFG in this list, but other implementations might have more than one.}
The idea of the \texttt{find\_flask\_route\_functions} was described in the previous paragraph.
A similar consideration has to be made for the framework in question and then implemented in a similar fashion.

After implementing a custom adaptor, \pyt{} will be able to analyse any application using this framework.
  

