\section{Handling Imports}
This section contains a description of how imports are handled when processing a file.
Handling imports is essentially just coupling called functions, class instantiations and variables to the correct definition in another file.
As such it is not theoretically interesting.
However, it is a requirement for being able to run \pyt{} on real projects, which is important because we want to evaluate the security of complete web applications.

\paragraph{The import statements}
The two ways of importing in Python, import and import-from, are presented in \cref{python:import} and these two ways need to be handled separately.
When referencing a function in a module imported with the \texttt{import} statement the function needs to be prefixed with the name of the module while this is not the case for \texttt{import-from}.


The file that is given as input to \pyt{} is used as the entry point.
First, the \texttt{project\_handler} module is used to get what we call local modules and project modules.
A local module is a Python file that is in the same folder as the file.
A project module is a Python file that is in the same project as the file.
The project modules are used to resolve the location of modules placed in another directory.

To illustrate the above consider the following example project structure below.

\hfill
\dirtree{%
.1 /.
.2 car\_app.
.3 app.py.
.3 models.py.
.3 user.
.4 forms.py.
.4 views.py.
}  
\hfill

Consider the entry point being the \textit{app.py} file.
The \texttt{project\_handler} module will construct a list of project modules containing the following modules: [\textit{car\_app.app.py, car\_app.models.py, car\_app.user.forms.py, car\_app.user.views.py}].
And a list of local modules containing the modules that are in the same folder as the entry file: [\textit{app.py, models.py}].

\paragraph{The module definitions stack}
Using the \texttt{ast} module, described in \cref{library:ast}, the abstract syntax tree(AST) is generated for the entry point.
When visiting an import or import-from node the module and its definitions are loaded.
A definition in this context is a definition of a variable, function or class.
A file being loaded usually also contains imports.
In order to control the different import contexts, a module definition stack is built.
Each entry in the stack contains a list of loaded definitions for one file.

To illustrate consider these two modules:

\begin{lstlisting}[style=python, caption={A module that defines a function and two classes called \texttt{a}}, label={import:definition_module}]
  def fuel(km):
      ...

  class Car():
      ...

  class Owner():
      ...
\end{lstlisting}
\begin{lstlisting}[style=python, caption={A module called \texttt{b} importing the above module \texttt{a}}, label={import:import_module}]
  ...
  
  import a
  
  ...

  c = a.Car()
\end{lstlisting}

First the AST for the entry file is created in this case we say its module \texttt{b}.
Then an entry in the stack created for \texttt{b}.
So the stack contains one module definitions: stack = [module\_definitions<b>].

The AST is visited and when an import statement is visited a new module is visited and a new module definitions is created and added to the stack.
Visiting the import of \texttt{a} results in the following stack: stack = [module\_definitions<b>, module\_definitions<a>]

The definitions in the \texttt{a} list are: \texttt{foo}, \texttt{Car} and \texttt{Owner}.
These are added to the list of module \texttt{a}: module\_definitions<a> = [\texttt{foo}, \texttt{Car}, \texttt{Owner}].
Additionally they are added to the list of module \texttt{b} with names prepended with an \texttt{a}:  module\_definitions<b> = [\texttt{a.foo}, \texttt{a.Car}, \texttt{a.Owner}].

This is important so that when one of the imported definitions later is used in \texttt{b} the list of \texttt{b} contains the correct definition.

Every time there is an import a new list is added to the stack.
When the import is done the stack is popped.
This process handles the names for different modules and nested imports.
