\section{Handling Imports}
This section contains a rationale of how the imports are handled.
The two way of importing files, import and import-from, are presented in \cref{python:import} and these two way need to be handled separately.
They are handled separately in the way that the naming for import is prefixed by the name of the module and this is not the case for import-from.


The file that is given as input to \pyt{} is used as the entry point.
First of the \texttt{project\_handler} module is used to get local modules and project modules.
A local module is a Python file that is in the same folder as the file.
A project module is a Python file that is in the same project as the file.
The projects entry point is the folder the file is in or if another project root is specified.

To illustrate the above consider the following example project structure below.

\dirtree{%
.1 /.
.2 car\_app.
.3 app.py.
.3 models.py.
.3 user.
.4 forms.py.
.4 views.py.
}

So lets say that the entry point is the \textit{app.py} file.
This will result in a list of project modules containing the following modules: [\textit{car\_app.app.py, car\_app.models.py, car\_app.user.forms.py, car\_app.user.views.py}].
And a list of local modules containing the modules that are in the same folder as the entry file: [\textit{app.py, models.py}].


