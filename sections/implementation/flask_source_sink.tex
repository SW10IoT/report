\subsection{Taint Analysis}
In order to find dangerous flows in a program we need to know what expressions introduce user input and what expressions may not receive input.
We also need to be able to detect when the program actually makes such input harmless.
The theoretical background for this was discussed in \cref{theory:taint}

\begin{lstlisting}[style=python, caption={A simple vulnerable program}, label=simple_sink_source]
x = input()
print('this is dangerous!')
eval(x)
\end{lstlisting}

The program in \cref{simple_sink_source} is vulnerable to a user executing arbitrary code.
\texttt{input()} is a source and \texttt{eval()} is a sink.
The result of \texttt{input()} is being handed over to \texttt{eval()} without any sanitisation.
Imagine this program being served on a server which also contains confidential information.
An attacker would possibly be able to retrieve this confidential information from the server.

\subsection{Sources, Sinks and Sanitisers in Flask}
Flask is a web framework, so it uses some different means of user input than the \texttt{input()} method.
This section will cover a selection of these in order to have a basis for performing an analysis of a Flask application.

\paragraph{Sources}
In \cref{flask:intro} the \texttt{request} object was introduced.
This object is used to access the input provided by the user through interaction with the page.

\subparagraph{Query parameters}
Query parameters submitted to the URL of the format 'domain.com?query=parameter' can be accessed with the \texttt{request.args.get(key, default)} method.
This method takes the key to retrieve a value from and a default value as parameters.
If this methods returns a value that is different from the default, it is potentially dangerous.

\subparagraph{Form content}
Input submitted through form fields can be accessed with the \texttt{request.form[key]} attribute.
This attribute is a dictionary that maps a form field to its value.
Values retrieved from this dictionary are potentially dangerous.

\paragraph{Sinks}
User input can arrive at a wealth of methods which can be regarded as a sink.
We explored three vulnerabilities in \cref{security_vulnerabilities} which we would like to be able to detect.
The following is the sinks that can possibly enable the chosen vulnerabilities.
These are obviously only a small selection of all possible sinks.

\subparagraph{Homemade templates}
When utilising the template engine of Flask many security concerns are ruled out.
An eager developer that is too impatient to read the documentation for \emph{Jinja2} may resort to a solution that works, but is subpar regarding security.
One example of this could be to use \texttt{string.replace(old, new)} instead of the template engine.
The developer creates his own template format, and replaces placeholders with user input.
It works, but if the input does not get sanitised, the user can inject javascript into the page, see \cref{vulnerabilities:xss}.

The input passed to \texttt{string.replace()} can be sanitised using the \texttt{Markup} class provided by Flask.
\texttt{Markup} is used to represent HTML markup and marks them as being safe.
It contains an \texttt{escape} function that escapes HTML tags and thereby sanitises the input.
Manipulating a markup object with string interpolation also escapes all values.
An example of escaping user input can be seen on \cref{escape} where the \texttt{Markup.escape} method is used to sanitise the input before inserting it with the \texttt{string.replace()} method.

\begin{lstlisting}[style=python, caption={An example of escaping user input}, label={escape}]
  html = open('templates/example1.html').read()
  unsafe = request.args.get(param)

  sanitised = Markup.escape(unsafe)
  
  resp = make_response(html.replace('{{ param }}', sanitised))
\end{lstlisting}


\subparagraph{Database access}
The recommended way to work with a database in Flask is by using a package called SQLAlchemy\cite{sqlalchemy}.\cite{sqlalchemy_docs}
This package gives the developer the option to setup and/or communicate with a database.
In the following two examples of SQL injection, described in \cref{vulnerabilities:injection}, will be presented.
The complete source code of the web application can be found in \cref{sql_injection_example}.
The examples both print out the result of the SQL queries to the console.
In a real application their would be some kind of UI where it is presented to the user.


The first example, \cref{sqli_raw}, shows that the address ``/raw'' takes a parameter called \texttt{param}.
This parameter is then executed directly on the database.
The result is printed to the console.
The problem here is that the user is able to execute every query he wants.
But SQLAlchemy prevents from executing several SQL commands and from changing the database by for example deleting all of it.

\begin{lstlisting}[style=python, caption={An example of executing a raw sql query without sanitising.}, label={sqli_raw}]
@app.route('/raw')
def index():
    param = request.args.get('param', 'not set')
    result = db.engine.execute(param)
    print(User.query.all(), file=sys.stderr)
    return 'Result is displayed in console.'
\end{lstlisting}

The second example, \cref{sqli_filter}, shows how the parameter of the filter function is not sanitised.
The address ``/filter'' takes a parameter, \texttt{param}, as input.
This \texttt{param} is used directly in a filter function on the database.
If the value of the \texttt{param} is ``2 or 1 = 1'' the filter function will return all users because the \texttt{param} always evaluates to true.
\begin{lstlisting}[style=python, caption={An example of inserting a raw sql parameter without sanitising it.}, label={sqli_filter}]
@app.route('/filter')
def filter():
    param = request.args.get('param', 'not set')
    Session = sessionmaker(bind=db.engine)
    session = Session()
    result = session.query(User).filter("username={}".format(param))
    for value in result:
        print(value.username, value.email)
    return 'Result is displayed in console.'
\end{lstlisting}

\subparagraph{Filesystem access}
Files can be sent to the client with the \texttt{send\_file(filename)} method.
This method takes a file from the filesystem and sends it to the user.
If the filename parameter is set by user input, the user can possibly traverse the filesystem of the server, see \cref{vulnerabilities:traversal}.
This can be sanitised by disallowing the use of '..' in the variable passed to the filename parameter.
