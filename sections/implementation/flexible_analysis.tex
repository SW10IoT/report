\section{Flexible analysis}\label{impl:flexible}
As mentioned in \cref{impl:overview} the fixed point analysis is implemented so that the analysis can be changed.
For our purposes the \emph{reaching definitions} analysis is sufficient -- it provides the answers we need to locate the security vulnerabilities in web applications.

However, one could imagine improving \pyt{} by utilising other analyses that could help identify the vulnerabilities.
In such a case \pyt{} has been implemented so it can be easily expanded with another analysis.
The liveness analysis presented in \citet[p.~19]{schwartzbach} has been implemented in order to exemplify the process of implementing new analyses.
This process will be explained in the following.

\subsection{Implementing Liveness}
The complete implementation is presented in \cref{appendix:liveness}.
The implementation corresponds to the definition presented in \citet[p.~20]{schwartzbach}.
The \texttt{LivenessAnalysis} class inherits from \texttt{AnalysisBase} which contains logic for annotating the CFG with extra information.
In the case of the liveness analysis we need information about the \texttt{vars(E)} of each node.
The deduction of these are defined as a visitor \texttt{VarsVisitor}, see \cref{cfg:visitor} which visits an AST node and finds the variables as defined in \citet{schwartzbach}.

\texttt{LivenessAnalysis} contains the constraints of the liveness analysis defined in \texttt{fixpointmethod}, which is shown in \cref{liveness_impl_fixpointmethod}.
This method is used by the fixed point analysis to apply the constraints on each node.
This method uses a helper \texttt{JOIN} which corresponds to the JOIN function in \citet{schwartzbach}, and the variables found by \texttt{VarsVisitor}.

\begin{lstlisting}[style=python, caption={Implementation of liveness analysis - \texttt{fixpointmethod}}, label={liveness_impl_fixpointmethod}, escapeinside={(*@}{@*)}, firstnumber=24]
    def fixpointmethod(self, cfg_node):
        """Setting the constraints of the given cfg node
        obeying the liveness analysis rules."""
    
        # if for Condition and call case: Join(v) u vars(E).
        if cfg_node.ast_type == Compare.__name__ or
           cfg_node.ast_type == Call.__name__:
            JOIN = self.join(cfg_node)
            # set union
            JOIN.update(self.annotated_cfg_nodes[cfg_node])  
            cfg_node.new_constraint = JOIN

        # if for Assignment case: Join(v) \ {id} u vars(E).
        elif isinstance(cfg_node, AssignmentNode): 
            JOIN = self.join(cfg_node)
            # set difference
            JOIN.discard(cfg_node.ast_node.targets[0].id)
            # set union
            JOIN.update(self.annotated_cfg_nodes[cfg_node])  
            cfg_node.new_constraint = JOIN

        # if for entry and exit cases: {}.
        elif cfg_node.ast_type == "ENTRY" or
             cfg_node.ast_type == "EXIT":
            pass
        # else for other cases.
        else:
        cfg_node.new_constraint = self.join(cfg_node)
\end{lstlisting}

With this implementation the \pyt{} fixed point algorithm implementation can perform liveness analysis on an arbitrary piece of python source code.
The \pyt{} vulnerability checker could likewise be expanded with any analysis that could aid the detection of security vulnerabilites.
