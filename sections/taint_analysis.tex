\section{Taint Analysis}
The following is based on \citet[Part 3]{schwartz2010all}.
Taint analysis is used for tracking the information between sources and sinks.
If some data comes from an untrusted or tainted source it is a tainted.
All other data is untainted.
Which sources introduce taint are defined by the user of the analysis.

If a taint analysis is marking too much data as taint it is called \textit{overtainting}.
If a taint analysis is marking too little data as taint it is called \textit{undertainting}.
If the analysis is \textit{precise} it means there is no over- or undertainting.
Undertainting leads to missing some vulnerabilities and overtainting leads to having some false positive vulnerabilities.

Because finding all vulnerabilities is critical it is common to use the approach of overtainting.
This can lead to a problem called \textit{taintspread} which is when more and more data is tainted and that with less precision.
To help with that, one can use \textit{sanitisers} to untaint data.
