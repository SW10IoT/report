\section{Fixed-point}
For the static analysis of a program we need a theorem that can find the best approximation of some dataflow problem.
Because we will generally consider problems that are undecidable, see \cref{theory_intro}, we need a framework that overestimate the answer, and thus makes an approximation.

This framework will be based on the fixed-points theorem which states the following (quoted from \citet[p.~13]{schwartzbach}):

\begin{definition}{Fixed-point theorem}
In a lattice L with finite height, every monotone function $f$ has a unique least fixed-point defined as
\[ fix(f) = \bigsqcup_{i \ge 0} f^i(\bot) \]

for which $f(fix(f)) = fix(f)$
\end{definition}

\noindent
The proof of this theorem can be found in \citep[p.~13]{schwartzbach}.

This theorem enables us to find an approximation to an undecidable problem by walking up the lattice until a fixed-point is reached.
This computation has been illustrated in \cref{lattice_walk} where the analysis starts at $\bot$ and end in the fixed-point which is the approximation to the problem at hand.

\begin{figure}
\begin{center}
\includegraphics[width=0.2\textwidth]{figures/fixed-point_walk}
\end{center}
\caption{A walk through the lattice, starting a $\bot$ and ending in the fixed-point}
\label{lattice_walk}
\end{figure}
