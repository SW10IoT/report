\section{Flask}\label{flask:intro}
Flask is a framework for developing web applications in python\citep{flask_official}.
Its goal is to be minimal without compromising functionality.
It is extensible and flexible, so components like database, template engine and form validation can be chosen by the developer.

The minimal nature of Flask makes it possible to write a web page in a very small amount of code.
The following program, see \cref{minimal_flask}, creates a web server that serves a ``Hello World!'' page on the root.

\begin{lstlisting}[style=python, caption={Minimal Flask application}, label={minimal_flask}]
from flask import Flask
app = Flask(__name__)

@app.route('/')
def hello_world():
    return 'Hello World!'

if __name__ == '__main__':
    app.run()
\end{lstlisting}

The Flask framework is imported and a function is defined with the \texttt{app.route} decorator \todo{beskriv decorators i python intro, \#31} which tells the framework where to serve the page.
The function returns the string ``Hello World!'' which is shown on the web page.

The following will present the Flask functionality that will be used in this project.
The descriptions will be based on the \citet{flask_docs}.

\paragraph{Routing}
The \texttt{app.route} decorator is used to bind a function to a URL.
The parameter provided binds the function to the relative path.
Example paths are '/' for the root and '/hello\_world' for a sub-page.
The example in \cref{minimal_flask} binds the function \texttt{hello\_world()} to the  url \texttt{hostname/}

\subparagraph{HTTP methods} Another parameter for the \texttt{app.route} decorator is the methods allowed.
This parameter enables other HTTP methods than the default GET.
An example of enabling the POST method can be seen on \cref{methods}.

\begin{lstlisting}[style=python, caption={Enabling POST requests for a login form}, label={methods}]
@app.route('/login', methods=['GET', 'POST'])
def login():
    if request.method == 'POST':
        do_the_login()
    else:
        show_the_login_form()
\end{lstlisting}

\paragraph{Requests}
Interaction with the incoming request happens through the \texttt{request} object.
This object contains all attributes of the request such as arguments from the query string, form data from POST requests and uploaded files.
\Cref{query_string} shows a very simple use of the query string.
The query string 'name' is retrieved with the default value 'no name' and assigned to \texttt{param}, which is then returned to the user.

\begin{lstlisting}[style=python, caption={Simple use of the query string}, label={query_string}]
@app.route('/query_param')
def query_param():
    param = request.args.get('name', 'no name')
    return name
\end{lstlisting}

\paragraph{Responses}
When returning from a function that will be rendered as a web page, Flask provides several possibilities that makes this process flexible.
For example a returned string will automatically be converted to a valid HTML page.
Special functions for creating responses exist, such as \texttt{send\_file(path, name)} which sends a file to the client.
HTML files can be rendered with \texttt{make\_response()}, and more complex pages can be constructed with the template engine.
The template engine will be explained in the following.

\subparagraph{Templates}
Flask uses a template engine called \emph{Jinja2}, which helps the developer keep the application secure.
The template engine escapes any dangerous user inputs without the developer having to consider it.

\begin{lstlisting}[style=python, caption={Rendering a template that displays the name parameter}, label={template}]
from flask import render_template
  
@app.route('/hello/<name>')
def hello(name=None):
    return render_template('hello.html', name=name)
\end{lstlisting}

\begin{lstlisting}[style=python, caption={The Jinja2 template used by the hello example}, label={template_html}]
<!doctype html>
<title>Hello from Flask</title>

<h1>Hello {{ name }}!</h1>
\end{lstlisting}

\Cref{template} shows a simple example of a template being rendered.
The \texttt{render\_template} function takes the template displayed in \cref{template_html} and replaces the name variable with the name entered in the URL.
The template engine handles all escaping, so a malicious user cannot compromise the page through the URL.

\subparagraph{The Response object}
Sometimes a page is more complex than just rendering a template and replacing variables with the appropriate values.
Web pages have response codes to indicate errors, headers that define the parameters of the request and cookies to keep track of the user.
In order to work with these aspects we need to get hold of the response before sending it to the client.
This is done with the \texttt{make\_response()} method.
The usage of the \texttt{make\_response()} method is shown in \cref{make_response} where a handler for 404 errors is defined.
Instead of showing the browser's default 404 error, we want to show our own error.
\texttt{make\_response()} takes a parameter for setting the status code, so creating the custom 404 handler just renders a template and sets the error code to 404.
The \texttt{make\_response()} method returns a response object, which can then be manipulated by for example add a header, or a cookie with \texttt{set\_cookie(name, value)}.

\begin{lstlisting}[style=python, caption={Using make\_response to create a custom 404 handler}, label={make_response}]
@app.errorhandler(404)
def not_found(error):
    resp = make_response(render_template('error.html'), 404)
    resp.headers['X-Something'] = 'A value'
    return resp
\end{lstlisting}

\paragraph{SQLAlchemy}
SQL Alchemy is a toolkit for operating on SQL databases directly from Python.
Its goal is to provide ``efficient and high performing database access, adapted into a simple and Pythonic domain language''\citep{sqlalchemy_official}.

The code displayed on \cref{sqlalchemy_model} shows how the connection to a database is created.
The URI of the database is provided to the config attribute of the flask application, and a database object is created.
This object contains a \texttt{Model} class that can be used to declare the model.
In the example, a \texttt{User} model is declared.

\begin{lstlisting}[style=python, caption={A SQLAlchemy database model}, label={sqlalchemy_model}]
from flask import Flask
from flask_sqlalchemy import SQLAlchemy

app = Flask(__name__)
app.config['_DATABASE_URI'] = 'sqlite:////tmp/database.db'
db = SQLAlchemy(app)


class User(db.Model): 
    id = db.Column(db.Integer, primary_key=True)
    username = db.Column(db.String(80), unique=True)
    email = db.Column(db.String(120), unique=True)

    def __init__(self, username, email):
        self.username = username
        self.email = email
\end{lstlisting}

Now that the database model is created we can populate it.
In \cref{sqlalchemy_use} the usage of a SQLAlchemy database is shown.
Between \cref{populate_start} and \cref{populate_end} Users are created and inserted into the database.
Afterwards, in \cref{query}, the database is being queried, first to show all Users, and afterwards to show only the first User named 'admin'.

\begin{lstlisting}[style=python, caption={Populating the database and querying the content}, label={sqlalchemy_use}, keywordstyle=\color{black}, escapeinside={(*@}{@*)}]
admin = User('admin', 'admin@example.com') (*@ \label{populate_start} @*)
guest = User('guest', 'guest@example.com')

db.session.add(admin)
db.session.add(guest)
db.session.commit() (*@ \label{populate_end} @*)

users = User.query.all() (*@ \label{query} @*)
admin = User.query.filter_by(username='admin').first()
\end{lstlisting}

