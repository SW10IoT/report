\section{Control Flow Graph}\label{control_flow_graph}
If one wants to analyse a program and the analysis is flow sensitive one can use a Control Flow Graph(CFG) to track the flow of the program.
An analysis flow sensitive means that it is significant in which order pieces of the program are executed.
A CFG is simply just another representation of the program source.
It is a directed graph where a node is a specific place in the program and edges represent the possible options to which the program can execute to.
A CFG always consists of one entry node and one exit node, called \textit{entry} and \textit{exit}.
Given a node \textit{n} the set of predecessor nodes are denoted \textit{pred(n)} and the set of successors are denoted \textit{succ(n)}.

\todo{Ved ikke lige med opsætning af figurer}
The control flow structures in the Python programming language are presented in \cref{python:control_structures}.
The graphs presented in \cref{python:control_structures} are the corresponding CFGs to the examples presented.
\textbf{Note} that the \textit{else} clause is not represented as a note because it has no influence on the flow.
The flow is determined by the \textit{if} clause, the \textit{else} just offers more options.
