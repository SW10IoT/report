\section{Taint Analysis}
The following is based on \citet[Part 3]{schwartz2010all}.
Taint analysis is used for tracking the information between sources and sinks.
If some data comes from an untrusted or tainted source it is regarded as being tainted.
All other data is untainted.
Which sources introduce taint are defined by the user of the analysis.

If a taint analysis is marking too much data as tainted, the analysis is \textit{overtainting}, while an analysis that is marking too little data as tainted it is \textit{undertainting}.
If the analysis is neither over- or undertainting it is \textit{precise}. 
Undertainting leads to missing some vulnerabilities while overtainting leads to false positives.

Because finding all vulnerabilities is critical it is common to strive to overtainting rather than undertainting.
This can lead to a problem called \textit{taintspread} which is when more and more data is tainted and that with less precision.
To help with that, one can include \textit{sanitisers} in the analysis which untains the data.
