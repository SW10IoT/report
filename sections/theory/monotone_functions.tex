\section{Monotone Functions}
A function $f: L \to L$ is monotone when $\forall x,y \in S : x \sqsubseteq y \Rightarrow f(x) \sqsubseteq f(y)$.
Being monotone does not imply that the function is increasing.
All constant functions are monotone.

If we look at the least upper bound and greatest lower bound as functions they are monotone in both arguments.
This means that we can apply the least upper bound function on two arguments and be certain that it will never decrease.
This is an important requirement for the fixed-point theorem which will be presented in the next section.
