\section{The Fixed-point Algorithm}\label{fixed_point_algorithm}
A CFG has been defined in \cref{control_flow_graph} and a lattice has been defined in \cref{lattice}.
Given a CFG and lattice a fixed-point algorithm can be defined.
The following is from \citet{schwartzbach}.


Operating on a CFG that contains the nodes $\{ v_1, v_2, \dots, v_n \}$ we work with the lattice $L^n$.
Assuming that the node $v_i$ generates the dataflow equation $\llbracket v_i \rrbracket = F_i ( \llbracket v_1 \rrbracket, \dots, \llbracket v_n \rrbracket)$, the equations of all the nodes can be combined in a function $ F: L^n \rightarrow L^n$.

\[ F(x_1, \dots, x_n) = (F_1(x_1, \dots, x_n), \dots, F_n(x_1, \dots, x_n)) \]

This combined function resembles the function presented in \cref{theory:fixedpoint:systemsofequations}, and it can indeed be solved by fixed-points.
A naive algorithm for solving such a function is presented in \citet{schwartzbach}:

\begin{algorithm}
  \caption{The naive Fixed-Point algorithm as presented in \citet[p.18~]{schwartzbach}}\label{fixed-point_algo}  
  \DontPrintSemicolon
  $x = (\bot, \dots, \bot)$ \\
  \Do{$ x \ne t $}
         {
           $t = x$\\
           $x = F(x)$
         }
\end{algorithm}

As $L^n$ is finite and assuming that the constraints that make up $F(x)$ are monotonous, the fixed-point algorithm will always terminate according to the Fixed-point theorem, see \cref{fixed_point_definition}.

\todo{if we use the more advanced algorithms, write them here}
