\section{Analysis Extension}\label{theory:extension}
The previously described algorithm is able to detect that a value flows from a source to a sink.
Unfortunately this method is insufficient in finding all vulnerabilities.
When a value from a source is assigned to a variable \textit{s}, this variable is not necessarily the one being used when arriving at a sink.
Other variables can be assigned to values that derive from \textit{s}, or \textit{s} can be reassigned to a modified version of \textit{s}.

These two problems arise from using the method presented in \cref{theory:reaching}.
Two small examples of the problems can be seen in \cref{ext:reassign} and \cref{ext:assign}.
Here the method \texttt{source} is an arbitrary method that introduces taint while \texttt{sink} is an arbitrary sensitive sink.

\begin{figure}[h]

  \begin{subfigure}[b]{0.45\textwidth}
    \begin{lstlisting}[style=python]
s = source()

s = 'input: ' + s

sink(s)
    \end{lstlisting}
    \caption{Reassigning a tainted variable with a modification of the tainted value}
    \label{ext:reassign}
  \end{subfigure}
  ~~
  \begin{subfigure}[b]{0.45\textwidth}
    \begin{lstlisting}[style=python, numbers=none]
s = source()

user_input = 'input: ' + s

sink(user_input)
    \end{lstlisting}
    \caption{Assigning a derived value of the tainted variable}
    \label{ext:assign}
  \end{subfigure}
\end{figure}

In neither of the two examples it will be possible to detect the vulnerability with the method from \cref{theory:reaching}, because the flow is cut off by the assignments.
To fix this we have introduced two extensions to the method.

\subsection{Propagation of Reassignments}
In order to maintain the tainted variable \textit{s} 'alive' we have modified the assignment dataflow constraint presented in \cref{theory:reaching}.
We want to let both the original assignment as well as the reassignment live in the analysis.
First we check if the current assignment is a reassignment.
An assignment is a reassignment if the variable assigned to also is part of the right hand side.
If it is a reassignment, we omit the arrow function of the original rule, else the rule is as defined originally.
The extended constraint rule is shown below:

\begin{align*}
  \constraint{v} = &\text{ if v is a reassignment:}\\
  &\phantom{==}JOIN(v) \cup \{ v \}\\
  &\text{ else:}\\
  &\phantom{==}JOIN(v) \downarrow id \cup \{ v \}\\
\end{align*}

Now both the original assignment and the reassignment will propagate to the sink.
This modification will still guarantee that the analysis will terminate, because the analysis will still only walk up the lattice.
This is true because the added case only adds a node to the constraint, which causes the analysis to walk up in the lattice.

\subsection{Assignment of Variable Derivations}\label{ext:derivation}
To be able to detect derivations of variables as being vulnerabilities, we need to keep track of assignments to these derivations.
This will be done when finding sources in the CFG.
Each time a source is found, the subsequent nodes will be checked for assignments of derivations of this source node.
Nodes that fit this description will be saved together with the source node as a 'secondary' source node.
When checking for pairs of sources and sinks these assignments will also be considered as sources.
