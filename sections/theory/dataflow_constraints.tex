\section{Dataflow Constraints}
For analysing a CFG of a program we need to be able to describe the relationship between the individual nodes.
Some analyses track variable content while others track use of certain expressions.
Similarly, some analyses need information from its successors while others need information from its predecessors.

To describe these different analyses we annotate each node with a \emph{dataflow constraint}.
Dataflow constraints relate the value of a node to its neighbouring nodes and is denoted $\constraint{v}$.
A dataflow analysis consists of a number of dataflow constraints that each defines the relation of a construction in the programming language.

An example of a dataflow constraint could be conditions (from the liveness example in \citet{schwartzbach})
\[ \constraint{v} = JOIN(v) \cup vars(E) \]

where JOIN combines the constraints of the successors of $v$ and vars(E) denote the set of variables occurring in E.
