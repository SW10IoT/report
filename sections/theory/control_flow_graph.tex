\section{Control Flow Graph}\label{control_flow_graph}
The analysis we want to perform is flow sensitive, meaning that the order of the statements in the program influence the result of the analysis.
When performing a flow sensitive analysis, a Control Flow Graph (abbreviated CFG for convenience) can be used to describe the flow of the program.

A CFG is simply just another representation of the program source.
It is a directed graph where a node is a specific place in the program and edges represent the possible options to which the program can execute to.
A CFG always consists of one entry node and one exit node, called \textit{entry} and \textit{exit}.
Given a node \textit{n} the set of predecessor nodes are denoted \textit{pred(n)} and the set of successors are denoted \textit{succ(n)}.

\paragraph{Python control flow graphs}
CFGs for simple Python statements, such as an assignment, have an entry node, a node containing the assignment and an exit node.
The CFG of the sequence of two statements $S_1$ and $S_2$ are constructed by deleting the exit node of $S_1$ and the entry node of $S_2$ and \emph{gluing} the two resulting graphs together.
This process is depicted on \cref{cfg_sequence}.

\begin{figure}
  \begin{subfigure}[b]{0.49\textwidth}
    \center
    \includegraphics[scale=0.5]{figures/simple.pdf}
    \caption{A single statement}
  \end{subfigure}
  ~
  \begin{subfigure}[b]{0.49\textwidth}
    \center
    \includegraphics[scale=0.5]{figures/sequence.pdf}
    \caption{A sequence of two statements}
  \end{subfigure}
  \caption{Construction of a control flow graph for two sequential statements}
  \label{cfg_sequence}
\end{figure}

The CFGs of control flow structures in the Python programming language was presented in the figures provided in \cref{python:control_structures}.

The CFG for a complete program is produced by systematically combining the building blocks.
A CFG of the general example in \cref{theory:general_code_example}, is shown on \cref{theory:general_code_example_cfg}.

\begin{figure}
  \center
  \includegraphics[height=\textheight]{figures/general_example.pdf}
  \caption{The control flow graph of the general example, \cref{theory:general_code_example}.}
  \label{theory:general_code_example_cfg}
\end{figure}
