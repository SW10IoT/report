The amount of vulnerabilities in software is growing every day.
As new software and features gets published, potential vulnerabilities gets released.
The attacker has the advantage in terms of variety of attacks.
He can comfortably attempt each attack in his arsenal while the publisher frantically tries to patch up the breaches.
The publisher often finds out about a vulnerability when it is too late and important data has been stolen.
So the element of surprise is certainly also there.
As Bruce Schneier said:
\begin{quote}
`\textit{`Attackers are at an advantage in cyberspace – this will not always be true, but it’ll certainly be true for the next bunch of years – and that makes defense difficult.}''\cite{schneier_interview}  
\end{quote}

\paragraph{Vulnerabilities in web applications}
In regards to web application security there are popping up new technologies and old ones are evolving and developing new features.
These new technologies and features can be used by the attackers as well.
Therefore it is important to be updated on current and new technologies when developing anything to the web.\cite{web_security_importance}

The \citet{OWASP10} lists the most critical web application security flaws.
The list is produced by a broad spectrum of security experts. 
They recommend that each application is checked for these ten critical security flaws and each organisation gets aware of how to detect and prevent these flaws.

\paragraph{}
Considering the difficulty of these problems and the size of code bases in the average software project, it would be an advantage to have a tool that could help finding security vulnerabilities.
As both of us have an interest in Python development we decided to look into tools the could help developing secure Python web apps.
We encountered two Python web frameworks, Django and Flask, that both do their part in helping the developer.
But even though the framework helps the developer, it can still be circumvented, so we looked into tools that could analyse and find vulnerabilities in Python code.
When examining this type of tools we did not find anything that satisfied our needs, see \cref{}\todo{ref til afsnit der sammenligner python tools}.
This report will thus create a tool that aids detection of security flaws in Python web applications, focusing on the Flask web application framework.
