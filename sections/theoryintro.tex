Finding security flaws in a program cover a range of different techniques.
What we are interested in is to determine whether a dangerous input from the user is able to reach a place in the code where it can cause damage.


The bad news is, it turns out, that we will not be able to provide answers, but only appriximations to this problem.
This unfortunate fact is due to Rice's theorem which can be phrased as ``all interesting questions about the behaviour of programs are undecidable'' \citep[p.~3]{schwartzbach}.
It turns out that \emph{interesting questions} is a rather large group of problems, and our problem is part of this group.

We will therefore not be able to provide definitive answers for each input, but we will be able to provide an approximation that comes as close as possible.

Static analysis is the tool to use for these approximations.
Static analysis is technique used for answering this type of question by defining a number of rules which define the problem.
The solution to the problem will then be gradually approached by algorithm designed for this purpose.
The algorithm stops when it can not get closer to an answer.
This answer is then the approximation which can be used for further analysis.

This chapter will describe the theory used for building the static analysis engine used by PyT.
It is based on the lecture notes on static analysis in \citet{schwartzbach}
