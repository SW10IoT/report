\section{Parameter Passing}\label{python:parameter_passing}
This section describes how Python deals with parameter passing and is inspired by \citet{parameter_passing}, official documentation for this can be found at \citet{parameter_passing_official}.
This section is included as it is important to factor in when parameters are assigned in functions.

The most known parameter passing techniques are pass-by-reference and pass-by-value.
A short description of these two will lead up to an explanation of how Python is handling parameter passing.
Python is basically using pass-by-value but objects are passed by reference, this is the same as for instance in Java and C\#.
To illustrate the different approaches the following two functions are used, \cref{parameter_passing:reassign} and \cref{parameter_passing:append}.

\begin{lstlisting}[style=python, caption={Parameter passing: \texttt{reassign} function.}, label={parameter_passing:reassign}]
  def reassign(l):
      l = [1]
\end{lstlisting}

\begin{lstlisting}[style=python, caption={Parameter passing: \texttt{append} function.}, label={parameter_passing:append}]
  def append(l):
      l.append(1)
\end{lstlisting}

An abstract way of showing the internal representation will be used.
An example can be seen on \cref{parameter_example} where the variable \texttt{l} points at a list which is an object stored in memory as \texttt{[0]}.
In python a variable is just a name that points to some object in memory.
A name is illustrated as a box.
Assigning '1' to a variable 'k' and then reassigning it to '2' does not change the '1' in memory.
It just creates the '2' object in memory and makes the 'k' point at this object.\footnote{The garbage collector is removing the 1 if it is not used anymore.}

\ppv{l}{[0]}{A variable \texttt{l} pointing at a list \texttt{[0]}}{parameter_example}

\paragraph{Pass-by-reference}
Pass-by-reference is a parameter passing mechanism where the argument is directly passed into the function.
Consider passing \texttt{l = [0]} to the function \texttt{reassign}.
After the call the object \texttt{l} is changed to: \texttt{l = [1]}, visualised on \cref{parameter_reference_reassign}.

\ppv{l}{[1]}{The variable \texttt{l} pointing at the list \texttt{[0]} after calling function \texttt{reassign}}{parameter_reference_reassign}

This is because the variable is passed directly which means that the function is operating directly on the object.
The \texttt{append} function behaves similarly, but because append adds to the list the result is: \texttt{l = [0, 1]}, visualised in \cref{parameter_reference_append}.

\ppv{l}{[0, 1]}{The variable \texttt{l} pointing at the list \texttt{[0, 1]} after calling function \texttt{append}}{parameter_reference_append}


\paragraph{Pass-by-value}
The other well known parameter passign mechanism is pass-by-value, where the actual parameter is copied and passed into the function.
The actual parameter is copied and stored a new place in memory and the copy is passed into the function.
Given the object \texttt{l = [0]} passed as parameter to the function \texttt{reassign}, the copied object \texttt{l$'$} is changed to: \texttt{l$'$ = [1]}.
The original object \texttt{l}  remains the same as it is not manipulated by the function.

\begin{figure}[H]
  \ppvs{l}{[0]}{The variable \texttt{l} pointing at the list \texttt{[0]} after calling the functions \texttt{reassign}}{parameter_value_l}
  ~
  \ppvs{l$'$}{[1]}{The variable \texttt{l$'$} pointing at the list  \texttt{[1]} after calling function \texttt{reassign}}{parameter_value_lprime}
  \caption{When reassigning in pass-by-value, only the copied list is changed}
\end{figure}

A similar thing happens in the \texttt{append} function, where after calling the function \texttt{l = [0]} and \texttt{l$'$ = [0, 1]}.
To visualise see \cref{parameter_value_l} and \cref{parameter_value_lprime}, here it becomes clear that when we access \texttt{l} after the call nothing has changed.

\begin{figure}[H]
  \ppvs{l}{[0]}{The variable \texttt{l} pointing at the list \texttt{[0]} after calling the functions and \texttt{append}}{parameter_value_l}
  ~
  \ppvs{l$'$}{[0, 1]}{The variable \texttt{l$'$} pointing at the list \texttt{[0, 1]} after calling function \texttt{append}}{parameter_value_lprime}
  \caption{When appending in pass-by-value, only the copied list is changed}
\end{figure}

\paragraph{Pass-by-object-reference}
Pass-by-object-reference is the Python way of handling parameter passing.
In this parameter passing mechanism the argument is copied into a new variable local to the function, but both refer to the same object in memory.
Given the object \texttt{l = [0]} when calling the function \texttt{reassign}, a new variable \texttt{l$'$} is created that refers to the same \texttt{[0]} object in memory.
So \texttt{reassign} sets \texttt{l$'$ = [1]} but not \texttt{l} because \texttt{reassign} is not manipulating the object but only the name that is referring to it.
Calling \texttt{reassign} manipulates the variables and is visualised in \cref{parameter_object_reassign_l} and \cref{parameter_object_reassign_lprime}.

\begin{figure}[H]
  \ppvs{l}{[0]}{The variable \texttt{l} pointing at the list \texttt{[0]} after calling function \texttt{reassign}}{parameter_object_reassign_l}
  ~
  \ppvs{l$'$}{[1]}{The variable \texttt{l$'$} pointing at the list \texttt{[1]} after calling function \texttt{reassign}}{parameter_object_reassign_lprime}
  \caption{Pass-by-object-reference copies the variable, but points it at the same object as the original variable}
\end{figure}

When calling the \texttt{append} function the object is referenced and both \texttt{l} and \texttt{l$'$} are changed to \texttt{[0, 1]}.
Calling \texttt{append} manipulates the objects referenced by the variables.
This is visualised in \cref{parameter_object_append}.

\begin{figure}[h]
   \centering
\begin{tikzpicture}
  \node[draw, line width=0.05mm, minimum width=1cm,minimum height=1cm, outer sep=10pt] (mem1) at (1.75,5) {l};
  \node[draw, line width=0.05mm, minimum width=1cm,minimum height=1cm, outer sep=10pt] (mem2) at (1.75,3) {l$'$};  
  \node (result) at (3.5,4) {[0, 1]};
  \draw[<-] (result) -- (mem1);
  \draw[<-] (result) -- (mem2);
\end{tikzpicture}   
  \caption{The variable \texttt{l} and \texttt{l$'$} stored in memory as \texttt{[0, 1]} when calling function \cref{parameter_passing:append}.}
  \label{parameter_object_append}
\end{figure}
