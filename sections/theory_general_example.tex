\section{General Example}
In order to make this theory chapter more understandable an example is introduced which is used throughout this chapter.
The code for this example is shown in \cref{theory:general_code_example}.
The example corresponds to the example in \citet{schwartzbach} adapted to Python code.
\begin{lstlisting}[style=python, caption={The general code example used throughout the theory chapter}, label={theory:general_code_example}, escapeinside={(*@}{@*)}]
  x = input() (*@ \label{theory:general_code_example:input} @*)
  x = int(x)
  while x > 1:
      y = x / 2
      if y > 3:
          x = x-y
      z = x-4
      if z > 0:
          x = x / 2
      z = z - 1
  print(x)
\end{lstlisting}

The program is very simple, it takes some user input which is sent through a \texttt{while} loop that changes the three variables x, y and z, depending on two \texttt{if} statements.
The x variable is printed out as the last operation of the program.
Although it is simple it presents some interesting statements and scopes.
