\chapter*{Summary}
This report presents the static analysis too \pyt{} which has been created to detect security vulnerabilities in Python web applications, in particular applications built in the framework Flask.

The tool utilizes the monotone framework for the analysis.
An AST is built by the builtin AST library, and a CFG is built from the AST.
The resulting CFG is then processed so Flask specific features are taken into account.
A modified version of the reaching definitions algorithm is now run by the fixed-point algorithm to aid the finding of vulnerabilities.
Vulnerabilties are detected based on a definition file containing 'trigger words'.
A trigger word is a word that indicate where the flow of the program can be dangerous.
The detected vulnerabilities are in the end reported to the developer.

\pyt{} has been created with flexibility in mind.
The analysis can be either changed or extended so the performance of \pyt{} can be improved upon.
Also the Flask specific processing can be changed so other frameworks can be analysed without major changes to \pyt{}.

In order to test the abilities of \pyt{} a number of vulnerable applications was manufactured and \pyt{} was evaluated with these.
All the manufactured examples was correctly identified as being vulnerable by \pyt{}.

To test \pyt{} in a more realistic setting it was also run on 7 open source projects.
Here no vulnerabilities were found.
One of the projects was so big that \pyt{} spent very long on the analysis and was therefore terminated.

