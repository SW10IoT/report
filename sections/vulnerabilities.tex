\section{Security vulnerabilities}

To make a tool that finds security vulnerabilities in a program is a very broad and ambitious task.
In order to make the task more manageable we have picked three common types of vulnerabilities from the OWASP list of top 10 vulnerabilities in web applications \cite{OWASP10}.
The chosen vulnerabilities will in the following be described and code examples will be presented showing the vulnerability in practice.
These code examples will later be used for testing the application.

\paragraph{Injection}
An injection is an uncontrolled use of user input that reaches an interpreter.
This could be a SQL interpreter, the command line or the Python interpreter.
If an injection vulnerability exists the user can  send arbitrary commands to the server, and possibly access or alter data without authorization.
The typical workaround is to sanitize the input so only the desired characters will be accepted. \citep{OWASPTOP10PDF}

We have chosen to work with SQL injections which are a subgroup of injection attacks.

\subparagraph{SQL injection}
This description is based on \citet{sqlinjection}.SQL injections happen when an application creates a query partly from user input without sanitizing this input.
The user can escape the query and add an arbitrary query.
Depending on the system, the user can now change the database, read the database or in severe cases even execute system commands.

An implementation of a SQL injection can be found in \cref{appendix:sqli}.
Here a database is setup with the library SQLAlchemy which has protection against SQL injection built in.
The problem is that it also supports raw SQL queries, and this example only utilizes these.

In \cref{sql2} the author has prepared a SQL statement concatenated with a request parameter to find a particular user.
Because the resulting string is not sanitized, the user can input whatever he wants, and an input like
\[ name' ~ \text{OR} ~ 'a' = \text{'a} \]
will return all users in the database, which is of course not intended behaviour.
Note the missing apostrophe after the equality sign.
The query still needs to be valid, so the string is first closed with the string ``name'', and afterwards a tautology is appending in an OR construction.
This ultimately means that the statement sent to the SQL interpreter looks for all users instead of just a particular one.
\todo{mere dybde i eksemplet? Vis konsol output?}
\todo{html siderne ogsa i appendix?}

\paragraph{XSS}
Cross site scripting (abbreviated XSS) occur when a site takes untrusted data and sends it to other users without sanitization.
The description i based on \citet{crosssitescripting}.
One example could be a comment section where an evil user sends in a comment containing some JavaScript.
When other users see this comment, their browser runs this JavaScript, which can access cookies or redirect to a malicious site.
Again, this can be prevented by sanitizing user input that is stored.

A very simple example of a cross site scripting is presented in \cref{appendix:xss}.
In \cref{xss:request} some input is taken and it is inserted into the result page in \cref{xss:replace}.
A link to the vulnerable web page can be constructed and sent to a unknowing user who opens the link and gets important information stolen.
In this example the input is not saved permanently, but if it was saved as for example a comment, a malicious user could inject arbitrary JavaScript, which all subsequent users would execute in their browsers.

\paragraph{Path traversal}
Path traversal is when access to a resource gives unintended access to the file system of the server.
This description is based on \citet{pathtraversal}.
An example could be attaching an image to the page decided by some user input.
If the input is not validated an attacker will be able to traverse the file system with the parent directory element ('..'). 

An example of this vulnerability can be seen in \cref{appendix:path_traversal}.
Providing the request parameter as \emph{?image\_name=../file.txt} will open and show file.txt in the browser.
This could possibly result in database content or passwords to be accessed by intruders.


\todo{all these problems have a solution that are related to sanitization. Maybe we should write a section on sanitization and the different solutions that already exist}

\section{Detecting vulnerabilities}\label{vulnerabilities:detecting}
Common for these three security vulnerabilities is that some user input reaches a sensitive piece of code.
An SQL injection happens when user input reaches a database query
Cross site scripting happens when user input reaches some HTML.
Path traversal happens when user input reaches a piece of code that performs file system access.

To detect these vulnerabilities we need to create a tool that can connect information about user input and sensitive code to detect when the operations performed are dangerous.
