\section{Python Web Frameworks}
Before trying to remove security vulnerabilities in applications we needed a framework to focus on.
This section contains short descriptions of two of the available Python web frameworks.
The web frameworks were chosen from \citet{python_web_frameworks_list}.
There are two sets of Python web frameworks, full-stack and non-full-stack frameworks, one of each was chosen.
A full-stack framework is a framework that contains all you need to develop a web application.
A non-full-stack framework is a framework which does not contain all packages to develop a complete web application.
Choosing either of these two categories is a matter of taste, full-stack framework are ready out of the box while non full-stack needs additional packages, but non-full-stack frameworks have the advantage that the developer can choose his preferred packages for the job.
First the Django web framework was chosen, which is a full-stack framework.
Django was chosen because it is one of the most popular web frameworks.
The second choice was the Flask micro web framework, one of the most popular non-full-stack frameworks.

\subsection{Django}
Django\cite{django_cite} is a web framework that contains all you need to make a complete web application.
Django is using the Model-View-Controller architecture pattern(MVC)\cite{leff2001web}.
It is built in a way that forces the developer to add functionality in a specific way.
This means that there is not a lot customiseability in the architecture of the project.

Django operates with an abstract concept of apps.
An app contains the following modules:
\begin{itemize}
\item Main module - where the app is starting to execute code.
\item Tests module - testing of the app.
\item Views module - visualisation of the app.
\item Urls module - maps urls to views.
\item Models module - models for instance from a database.
\item Apps module - nested apps.
\end{itemize}
A Django web application consists of a combination of apps.
The power of Django is the ease of reuse of an app as they are easily linked together using urls.

\subsection{Flask}
Flask\cite{flask_official} is a micro web framework.
Flask is highly customizable as you can chose your own architecture of your web application.
Also it is possible for instance to make your own form validation or use a form validation package that one wants.
Flask comes with the following features:
\begin{itemize}
\item Development server
\item Unit test support
\item REST support
\end{itemize}
The power of Flask is that the developer is able to customise everything.
