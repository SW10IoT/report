\section{Python}
This section will shortly give an overview of the Python programming language version 3.5.1.
This section is based on \citet{python_docs}.

The Python programming language was created by Guido van Rossum in the early 1990s and is to this day the main author.
As the language has grown there are obviously many more contributors and everyone can contribute as Python is Open Source.

\subsection{Control Structures}
The python programming language has three control structures, \texttt{if}, \texttt{while} and \texttt{for}.
In the following they are presented as simple examples and a figure showing the possible flow.

\paragraph{\texttt{If}}
The \texttt{if} control structure has several statements:
\begin{enumerate}
\item A simple \texttt{if} statement
\item An \texttt{if-else} statement
\item An \texttt{if-elif-else} statement
\end{enumerate}
The \texttt{if} and \texttt{elif} statements contain a condition which is evaluated to either true or false.
If the condition is true the body of the statement is evaluated if it is false the interpreter will jump to the code after the body.

\begin{figure}
  \centering
  \begin{subfigure}[b]{0.4\textwidth}
    \begin{lstlisting}[style=python, caption={Code example.}, label={python:if:simple:code}]
if True:
    x = 0
    \end{lstlisting}
  \end{subfigure}
  ~ %add desired spacing between images, e. g. ~, \quad, \qquad, \hfill etc. 
  %(or a blank line to force the subfigure onto a new line)
  \begin{subfigure}[b]{0.4\textwidth}
    \centering
    \includegraphics[scale=.5]{./figures/if.pdf}
    \caption{Possible flows.}
    \label{python:if:simple:flow}
  \end{subfigure}
  \caption{A simple \texttt{if} control structure containing one \texttt{if} statement.}
  \label{python:if:simple}
\end{figure}

Taking a look at \cref{python:if:simple} we can see a simple \texttt{if} structure containing only one \texttt{if} statement.
The code, \cref{python:if:simple:code}, is an \texttt{if} statement with a condition, \texttt{True}.
If the condition holds the body is evaluated and if not the program moves on to the next piece of code.
This flow can be seen on \cref{python:if:simple:flow}.


\begin{figure}
  \centering
  \begin{subfigure}[b]{0.4\textwidth}
    \begin{lstlisting}[style=python, caption={Code example.}, label={python:if:else:code}]
if True:
    x = 0
else:
    y = 0
    \end{lstlisting}
  \end{subfigure}
  ~ %add desired spacing between images, e. g. ~, \quad, \qquad, \hfill etc. 
  %(or a blank line to force the subfigure onto a new line)
  \begin{subfigure}[b]{0.4\textwidth}
    \centering
    \includegraphics[scale=.5]{./figures/if_else.pdf}
    \caption{Possible flows.}
    \label{python:if:else:flow}
  \end{subfigure}
  \caption{An \texttt{if} control structure containing an \texttt{if} and an \texttt{else} statement.}
  \label{python:if:else}
\end{figure}

A more advanced \texttt{if} control structure can be seen on \cref{python:if:else}.
This example consists of an \texttt{if} statement and an \texttt{else} statement.
The condition of the \texttt{if} statement is evaluated and if it holds the body of the \texttt{if} is evaluated if no the body of the \texttt{else} is evaluated.
These possible flows can be seen on \cref{python:if:else:flow}.


\begin{figure}
  \centering
  \begin{subfigure}[b]{0.4\textwidth}
    \begin{lstlisting}[style=python, caption={Code example.}, label={python:if:elif:code}]
if True:
    x = 0
elif False:
    y = 0
else:
    z = 0
    \end{lstlisting}
  \end{subfigure}
  ~ %add desired spacing between images, e. g. ~, \quad, \qquad, \hfill etc. 
  %(or a blank line to force the subfigure onto a new line)
  \begin{subfigure}[b]{0.4\textwidth}
    \centering
    \includegraphics[scale=.5]{./figures/if_else_elif.pdf}
    \caption{Possible flows.}
    \label{python:if:elif:flow}
  \end{subfigure}
  \caption{An \texttt{if} control structure containing an \texttt{if}, an \texttt{elif}, and an \texttt{else} statement.}
  \label{python:if:elif}
\end{figure}


And now the last possible statement in an \texttt{if} control structure the \texttt{elif} statement.
The \texttt{elif} statement has a condition like the \texttt{if} statement and if it holds its body is evaluated.
An example can be found on \cref{python:if:elif}.


\paragraph{\texttt{while}}
The \texttt{while} control structure has a condition which evaluates to true or false.
If the condition is true the body is executed and the condition is evaluated again, if it still holds the body is executed again.
This process continuous until the condition is false.
A \texttt{while} control structure can also have an \texttt{else} statement.
The body of the \texttt{else} statement is executed when the condition of the \texttt{while} statement is false.
