This report engaged in creating a tool for detecting security flaws in Python web applications.
Firstly, the background for the project was examined, including the Python programming language and the Flask web framework.
We then proceeded to manufacture some Flask applications with vulnerabilities.
The resulting applications would then be used to check if the tool was able to detect them.

The theory needed to build such a tool was then described.
The theory includes control flow analysis, the monotone framework and taint analysis.

The tool \pyt{} was then implemented with the aim of being flexible.
\pyt{} is specialised to find vulnerabilities in Flask using a modified reaching definitions analysis, but it is the intention that both aspects can be expanded on.
It is thereby possible to implement support for Django or use other analyses, without having to make major changes.

At last \pyt{} was evaluated.
\pyt{} was able to find all the manufactured vulnerabilities.
It was also able to run on open source Flask projects and did not find any vulnerabilities in these.
