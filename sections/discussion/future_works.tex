\section{Future works}

Should contain:
\begin{itemize}
\item Udvid med Django, pyt er parametriseret, så det er let at gå igang med
\item Udvid med andre analyser der forbedrer resultaterne
\item Limitiations
\item Optimering af pyt -hvordan handler vi ~500000 cfg nodes ..
\end{itemize}

\subsection{Better trigger word definitions}
Finding sources and sinks was a feature where the litterature was sparse.
Plenty of papers use the terms, but no one described a clever way to find them.
We therefore devised our own system.

The implemented system uses a text file with the trigger words written in a very simple syntax.
The labels of CFG nodes are compared with the trigger words, and if they contain one, they are marked as being either a source, a sink or a sanitiser.
Using this simple implementation we ran into problems like finding the trigger word in variable names.
The quick solution was to add a '(' to the trigger word, which fixed that single problem.

In further development the whole trigger word system should be reconsidered.
The definition should be more sophisticated, and the comparison should be more intelligent than just comparing in the label.

\subsection{More efficient fixed-point algorithm}
In this project efficiency has not been a priority, and for the most part is has not been a problem.
In the evaluation performed on real projects, described in \cref{evaluation:real}, this suddenly became a hurdle.
It turns out that big projects produce CFGs that are so large that \pyt{} simply does not terminate in a realistic amount of time.
It would therefore be a natural next step to attempt to improve this inefficiency.
As mentioned in \cref{fixed_point_algorithm} this project has not spent time examining alternative implementations of the fixed-point algorithm.
One suggestion would therefore be to attempt to implement the improved version of the algorithm, in order to get a faster analysis.
